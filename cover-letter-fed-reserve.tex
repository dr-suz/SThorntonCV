
\documentclass[11pt,a4paper,sans]{moderncv}        

\moderncvstyle{casual}                           
\moderncvcolor{blue}                               % color options 'blue' (default), 'orange', 'green', 'red', 'purple', 'grey' and 'black'
%\renewcommand{\familydefault}{\sfdefault}         % to set the default font; use '\sfdefault' for the default sans serif font, '\rmdefault' for the default roman one, or any tex font name
%\nopagenumbers{}                                  % uncomment to suppress automatic page numbering for CVs longer than one page

% character encoding
\usepackage[utf8]{inputenc}                       % if you are not using xelatex ou lualatex, replace by the encoding you are using
%\usepackage{CJKutf8}                              % if you need to use CJK to typeset your resume in Chinese, Japanese or Korean

% adjust the page margins
\usepackage[scale=0.75]{geometry}
%\setlength{\hintscolumnwidth}{3cm}                % if you want to change the width of the column with the dates
%\setlength{\makecvtitlenamewidth}{10cm}           % for the 'classic' style, if you want to force the width allocated to your name and avoid line breaks. be careful though, the length is normally calculated to avoid any overlap with your personal info; use this at your own typographical risks...

% personal data
\name{Suzanne}{Thornton}
%\title{Rutgers University}                               % optional, remove / comment the line if not wanted
\address{Philadelphia, PA}{USA}% optional, remove / comment the line if not wanted; the "postcode city" and and "country" arguments can be omitted or provided empty
\phone[mobile]{+1~(863)~370~9389}                   % optional, remove / comment the line if not wanted
%\phone[fixed]{+2~(345)~678~901}                    % optional, remove / comment the line if not wanted
%\phone[fax]{+3~(456)~789~012}                      % optional, remove / comment the line if not wanted
\email{thornton.suzy@gmail.com}                               % optional, remove / comment the line if not wanted
\homepage{dr-suz.github.io/}                         % optional, remove / comment the line if not wanted
%\extrainfo{additional information}                 % optional, remove / comment the line if not wanted
%\photo[64pt][0.4pt]{picture}                       % optional, remove / comment the line if not wanted; '64pt' is the height the picture must be resized to, 0.4pt is the thickness of the frame around it (put it to 0pt for no frame) and 'picture' is the name of the picture file
%\quote{Some quote}                                 % optional, remove / comment the line if not wanted

% to show numerical labels in the bibliography (default is to show no labels); only useful if you make citations in your resume
%\makeatletter
%\renewcommand*{\bibliographyitemlabel}{\@biblabel{\arabic{enumiv}}}
%\makeatother
%\renewcommand*{\bibliographyitemlabel}{[\arabic{enumiv}]}% CONSIDER REPLACING THE ABOVE BY THIS

% bibliography with mutiple entries
%\usepackage{multibib}
%\newcites{book,misc}{{Books},{Others}}
%----------------------------------------------------------------------------------
%            content
%----------------------------------------------------------------------------------
\begin{document}
	
	\recipient{Federal Reserve Bank of Philadelphia} 
	{Stress Test Retail Supervisory Modeling Team}
	\date{\today }
	\opening{Dear Hiring Team,}
	\closing{Sincerely,}
	\makelettertitle
	

% complex datasets R, Python, high performing cluster computer environment, learn software packages and database systems 

% work with team members, collecting and analyzing data, reviewing industry research, writing research briefs, presenting 
%I have also been working in and leading interdisciplinary teams to strategize the collection of data, analyze data, write research reports and present findings first as a graduate student statistical consultant, next as an advisor to undergraduate students, as a volunteer advisor to the US Census Bureau, and, most recently, as a statistical researcher working on projects within the Statistical Engineering Division of the National Institute of Standards and Technology.   
% become well informed on potential changes to regulations and supervisory policies 

% prepare and present recommendations to internal and external stakeholders 

% identify, develop, conduct research projects, and publish and present at academic conferences 

% ability to explain assessments on complex matters 
% experience working with large datasets 
% advanced analytical and quantitative skills 
% expert knowledge in statistics 
% advanced data analysis experience 
% strong analytical skills 
% present technical issues to tech and nontech audiences 
% team player with written and oral communication skills 
%Since graduate school, I have traveled the globe to present my research to audiences of varying statistical expertise. In addition to working in a governmental advisory capacity and teaching introductory and advanced undergraduate students, I have led educational programs for statistics instructors in collaboration with the University of Texas Austin DANA Center for Mathematics Pathways.  
% time management skills 
% self quality assurance 
%Passion for quality assurance and a growing awareness of my strengths and weaknesses that guides my own assessment of the quality of my work. Understanding priorities and deadlines. 
	%I began to apply my technical skills as a graduate student where I collaborated with a doctor from the Robert Wood Johnson hospital to develop a predictive model for drug-resistant epilepsy. As the sole statistical consultant on this project, I was responsible for data cleaning and exploration, model building and evaluation, and drawing proper conclusions from the data and model. This collaborative experience in developing data-driven solutions to a pressing medical issue resulted in a publication in the renowned journal \emph{Neurology}. After graduating, I published my dissertation research on an advanced computational inference technique called approximate confidence distribution computing while teaching at Swarthmore College. This research is related to approximate Bayesian computing and indirect inference, computational approaches for estimating parameters of complex models where traditional likelihood-based approaches fail. My approach is similar in that it relies upon synthetic data generation but is unique in that it prioritizes calibrated inferential conclusions. 
%incorporating ethical reasoning in stats classroom

	
%	My name is Suzanne Thornton and I am writing to apply for either the Senior Quantitative Analyst or Financial Economist position at the Federal Reserve Bank of Philadelphia (R-0000023772). Although my background is not in financial data analysis specifically, as a theoretical statistician with applied experience, I have the necessary skills and experience to excel in either position. I am providing this cover letter in an attempt to provide you with context to the experiences outlined in my resume. Because my published research agenda has thus far been outside the field of economics, I am happy to consider either position as a starting point to a fulfilling career promoting a strong economy and a stable financial system for my community and my country.    	
	
%	Since graduate school, I have worked with high performance computer clusters and R to analyze complex data with Bayesian and frequentist computational inference techniques. I have also traveled the globe to present my research to audiences of varying statistical expertise. I collaborated with and led interdisciplinary teams to strategize the collection of data, analyze data, write research reports, and present findings first as a graduate student, next as a faculty researcher and advisor to undergraduate students, then as a volunteer advisor to the US Census Bureau, and, most recently, as a statistical researcher working on projects within the Statistical Engineering Division of the National Institute of Standards and Technology (NIST).   

%	In the last couple of years, recognizing that I am particularly motivated by opportunities to innovate solutions to real world challenges, I have intentionally oriented my career towards statistical practice. As an affiliate of NIST, I've helped develop a novel Bayesian measurement error model to more accurately represent the relationship between nanoparticle size and loading capacity. I'm also currently working on developing guidelines for analyzing imputed clock data with an interdisciplinary team of scientists and statisticians. My publication history displays my ability to communicate highly technical analyses to a broad and varied audience. For example, my dissertation paper Approximate Confidence Distribution Computing (2023) in \emph{The New England Journal of Statistics in Data Science} is aimed at an academic statistical audience, my paper An Exploration of Parameter Duality in Statistical Inference (2023) in the \emph{Journal of the Philosophy of Science} is written for academic philosophers, and my earliest applied paper Development and Validation of a Predictive Model of Drug-Resistant Genetic Generalized Epilepsy (2020) was published in \emph{Neurology} and aimed at an audience of both medical practitioners and researchers. I also share these representative publications to provide evidence for the strength of my analytical skills and expert knowledge in statistics.  

%	My past experience in an academic environment has equipped me with opportunities to develop the critical skills applicable to this position. As a faculty, I enrolled in an online program where I began to formally develop my time management skills. I have continued to build upon these skills ever since through independent reading and regular planning practices.  My time as a faculty member also increased my awareness of my professional strengths and weaknesses, ultimately forging my ability to assess the quality of my work and manage priorities to meet deadlines. 

%	The values of collaboration, independence, and integrity have played an important role in my career thus far. In 2020, I was selected by the president of the American Statistical Association (ASA) to lead a working group on LGBTQ+ inclusion within the discipline. After this experience, I served on the ASA's LGBTQ+ Advocacy Committee and this year, I was proud to take on the role of committee chair. My most enjoyable work experiences thus far have been in interdisciplinary team settings where a diverse group of individuals work together towards shared goals. I have received positive feedback from these endeavors, not the least of which was overwhelmingly positive student feedback in the Spring semester of 2020, when our college courses abruptly switched to a virtual format. As an instructor, I also developed coursework that integrated the American Statistical Association's Guidelines for Ethical Practice and have several publications on ethical considerations for data collection and analysis.

%	My proficiencies as a leader in statistics have been recognized in different ways. Most recently, two internal NIST Building the Future grants were awarded to me and my co-authors for an exploration into standards for deepfake detection and for advancing statistical methods in clock metrology. While at Swarthmore, I was nominated by the ASA to join the National Advisory Committee to the US Census and as a graduate student, I received awards for my dissertation work. I am personally interested in the implications of regulations and governmental supervisory policies, especially as they pertain to my state and local economy. While this position offers me a new context for data analysis, I am confident that my independent interest will help me become quickly well-informed in these matters and ultimately provide another venue for my leadership capabilities. 

%	I am excited about the opportunity to begin a career at the Federal Reserve Bank of Philadelphia and to expand my knowledge and skillset by working with the dedicated members of the Stress Test Retail Supervisory Modeling Team. Please let me know if you have any additional questions for me and feel free to view my professional website (link in footer below) for additional information. Thank you for your time and consideration.\\
	
	
	I am writing to express my interest in the Senior Quantitative Analyst or Financial Economist position at the Federal Reserve Bank of Philadelphia. As a theoretical statistician with extensive applied experience, I am confident in my ability to contribute effectively to your team, despite my primary background being outside the field of financial data analysis. My academic and professional journey has equipped me with a strong foundation in statistical modeling, interdisciplinary collaboration, and problem-solving, all of which I believe align with the responsibilities and values of the Federal Reserve.

Throughout my career, I have specialized in using computational techniques to analyze complex data. With extensive experience in Bayesian and frequentist methods, I have worked with high-performance computing clusters and programming languages such as R and Python to develop and implement statistical models. This includes recent work with the National Institute of Standards and Technology (NIST), where I am developing a novel Bayesian measurement error model to represent nanoparticle size and loading capacity. I am also currently collaborating with a multidisciplinary team to create guidelines for analyzing imputed clock data. My diverse research experience, which spans both applied statistics and theoretical exploration, has enabled me to communicate complex analyses effectively to audiences with varying levels of expertise.

In addition to my technical skills, I have honed my ability to lead and collaborate in interdisciplinary settings. As a faculty researcher and advisor, I led teams to design studies, analyze data, and present findings, building strong time management, problem-solving, and communication skills. My leadership capabilities were further recognized through two NIST Building the Future grants focused on deepfake detection and advancing statistical methods in clock metrology. I am also proud to have served as chair of the American Statistical Association’s LGBTQ+ Advocacy Committee, demonstrating my commitment to ethical practices and inclusion in the field.

The values of collaboration, independence, and integrity are central to my professional identity. I am particularly drawn to this opportunity at the Federal Reserve Bank of Philadelphia because it offers a chance to apply my statistical expertise to real-world challenges in financial stability and economic policy. I am eager to contribute to your team’s work in promoting a strong and stable economy, and I believe my background in advanced statistical modeling, team leadership, and ethical data practices will be valuable assets to your organization.

Thank you for considering my application. I look forward to the opportunity to discuss how my skills and experiences align with the needs of your team. Please feel free to contact me at 863-370-9389 or thornton.suzy@gmail.com. For additional information, I invite you to visit my professional website at https://dr-suz.github.io/.
	
	\vspace{3mm}
	
	\makeletterclosing
	
\end{document}

