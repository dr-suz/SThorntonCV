
\documentclass[11pt,a4paper,sans]{moderncv}        

\moderncvstyle{casual}                           
\moderncvcolor{blue}                               % color options 'blue' (default), 'orange', 'green', 'red', 'purple', 'grey' and 'black'
%\renewcommand{\familydefault}{\sfdefault}         % to set the default font; use '\sfdefault' for the default sans serif font, '\rmdefault' for the default roman one, or any tex font name
%\nopagenumbers{}                                  % uncomment to suppress automatic page numbering for CVs longer than one page

% character encoding
\usepackage[utf8]{inputenc}                       % if you are not using xelatex ou lualatex, replace by the encoding you are using
%\usepackage{CJKutf8}                              % if you need to use CJK to typeset your resume in Chinese, Japanese or Korean

% adjust the page margins
\usepackage[scale=0.75]{geometry}
%\setlength{\hintscolumnwidth}{3cm}                % if you want to change the width of the column with the dates
%\setlength{\makecvtitlenamewidth}{10cm}           % for the 'classic' style, if you want to force the width allocated to your name and avoid line breaks. be careful though, the length is normally calculated to avoid any overlap with your personal info; use this at your own typographical risks...

% personal data
\name{Suzanne}{Thornton}
%\title{Rutgers University}                               % optional, remove / comment the line if not wanted
\address{Philadelphia, PA}{USA}% optional, remove / comment the line if not wanted; the "postcode city" and and "country" arguments can be omitted or provided empty
\phone[mobile]{+1~(863)~370~9389}                   % optional, remove / comment the line if not wanted
%\phone[fixed]{+2~(345)~678~901}                    % optional, remove / comment the line if not wanted
%\phone[fax]{+3~(456)~789~012}                      % optional, remove / comment the line if not wanted
\email{thornton.suzy@gmail.com}                               % optional, remove / comment the line if not wanted
\homepage{dr-suz.github.io/}                         % optional, remove / comment the line if not wanted
%\extrainfo{additional information}                 % optional, remove / comment the line if not wanted
%\photo[64pt][0.4pt]{picture}                       % optional, remove / comment the line if not wanted; '64pt' is the height the picture must be resized to, 0.4pt is the thickness of the frame around it (put it to 0pt for no frame) and 'picture' is the name of the picture file
%\quote{Some quote}                                 % optional, remove / comment the line if not wanted

% to show numerical labels in the bibliography (default is to show no labels); only useful if you make citations in your resume
%\makeatletter
%\renewcommand*{\bibliographyitemlabel}{\@biblabel{\arabic{enumiv}}}
%\makeatother
%\renewcommand*{\bibliographyitemlabel}{[\arabic{enumiv}]}% CONSIDER REPLACING THE ABOVE BY THIS

% bibliography with mutiple entries
%\usepackage{multibib}
%\newcites{book,misc}{{Books},{Others}}
%----------------------------------------------------------------------------------
%            content
%----------------------------------------------------------------------------------
\begin{document}
	
	\recipient{The Urban Institute} 
	{Data Governance and Privacy Practice Area}
	\date{\today }
	\opening{Dear Hiring Team,}
	\closing{Sincerely,}
	\makelettertitle
	
% The ideal candidate will bring subject matter expertise in statistical data privacy, a solid track record of conducting moderately complex statistical data privacy methods, strong written and verbal communication skills, and proficiency in R programming
% Job reqs 
% - Conducting research that harnesses technology, data science, and statistical methods, including data imputation, synthetic data generation, and formal privacy, to answer questions about practical social and economic policy topics.
% - Contributing to the development of proposals, responses to solicitations and other meaningful business cultivation activities.
% - Assembling and analyzing complex data, completing sophisticated programming tasks, and debugging programming problems using R, familiarity with Stata is a plus.
% - Analyzing data using techniques and platforms such as big data, machine learning, and cloud computing technology.
% - Developing open-source R packages for data science and statistical methods via distributed version control with Git and GitHub.
% - Using statistical knowledge to create insights from data.
% - Taking the initiative to suggest improvements or innovations of methods or processes.
% - Working with confidential administrative data that requires a background investigation.
% - Co-authoring research reports, briefs, blogs, presentations, visualizations, and other written materials for external audiences.
% - Supporting overall project management and team performance including directly managing assistants and analysts; monitoring project progress, budget, and schedule; and implementing strategies to successfully complete project tasks.
% - Communicating, educating, and translating complex data privacy concepts to various stakeholders.

% Enduring interest in domestic economic and social policy issues
	
	My name is Suzanne Thornton and I am writing to apply for the Research Associate position in Data Governance and Privacy (R-801591). The Urban Institute's dedication to contributing meaningful, evidence-based work to serve the public by expanding opportunities for all, reducing hardships for the most vulnerable, and strengthening the effectiveness of public policies aligns with my personal and professional values. As a life-long learner, I have come to believe that knowledge is second only to compassion. I wish to apply my expertise and grow my skills in a way that protects and supports human rights in our advanced age of technology. I am drawn towards this career opportunity in data governance and privacy because I believe these are critical areas for advancing policies that honor the rights of the public in our increasingly data-driven society.
%Indeed, I chose to pursue a career in Statistics not just so I could "play in other's backyards" as the prominent statistician John Tukey remarked, but so I could contribute towards building stronger, dependable, more fruitful backyards with the quantitative and technological tools at my disposal. 

%will work for a company that aims to provide growth opportunities for individuals and groups around the world in a trustworthy and sustainable way.  

%[highlight relevant technical expertise: data-driven solutions, up-to-date on research, brief example of significant technical project and how I contributed/impact I made]
	I began to apply my technical skills as a graduate student where I collaborated with a doctor from the Robert Wood Johnson hospital to develop a predictive model for drug-resistant epilepsy. As the sole statistical consultant on this project, I was responsible for data cleaning and exploration, model building and evaluation, and drawing proper conclusions from the data and model. This collaborative experience in developing data-driven solutions to a pressing medical issue resulted in a publication in the renowned journal \emph{Neurology}. After graduating, I published my dissertation research on an advanced computational inference technique called approximate confidence distribution computing while teaching at Swarthmore College. This research is related to approximate Bayesian computing and indirect inference, computational approaches for estimating parameters of complex models where traditional likelihood-based approaches fail. My approach is similar in that it relies upon synthetic data generation but is unique in that it prioritizes calibrated inferential conclusions. In the last couple of years, I have intentionally oriented my career towards statistical practice which I knew would afford me more opportunities to innovate solutions to real world challenges. As an affiliate of the National Institute for Standards of Technology (NIST), I've helped develop a novel Bayesian measurement error model to more accurately represent the relationship between nanoparticle size and loading capacity. I'm also currently working on developing guidelines for clock data time series imputation techniques with an interdisciplinary team of scientists and statisticians.
%Working with a mentor who encouraged me to develop my own understanding of the data and the challenges it presented, I began with an exploration of the data and of the prominent research on measurement error models. This convinced me that measurement error is a key component of the problem at hand and led to my acquiring new programming skills to experiment with different Bayesian methods for modeling this important nanoparticle relationship. A publication of our approach is in progress as we believe our model contributes a valuable methodology in areas of application ranging from environmental to medical.    
%[problem solving and innovation: specific example of a challenge I solved highlighting analytical approach and creativity] 
%incorporating ethical reasoning in stats classroom

	Urban's values of collaboration, equity, inclusivity, independence, and integrity have played an important role in my career thus far. In 2020, I was selected by the president of the American Statistical Association (ASA) to lead a working group on LGBTQ+ inclusion within the discipline. After this experience, I served on the ASA's LGBTQ+ Advocacy Committee and this year, I was proud to take on the role of committee chair. My priorities as a leader in these positions have been to foster an inclusive, welcoming environment where people feel empowered to contribute their perspectives and to explore their curiosity and their passions. My most enjoyable work experiences thus far have been in interdisciplinary team settings where a diverse group of individuals work together towards shared goals. I have received positive feedback from these endeavors, not the least of which was overwhelmingly positive student feedback in the Spring semester of 2020, when our college courses abruptly switched to a virtual format. As an instructor, I also developed coursework that integrated the American Statistical Association's Guidelines for Ethical Practice and have several publications on ethical considerations for data collection and analysis.
	%Each year as a faculty member at Swarthmore College, I also took on undergraduate research assistants and mentoring roles.
%[collaboration and leadership: highlight my ability to work effectively within teams, communicate complex ideas, and lead diverse groups towards shared goals] 
%ASA service and leadership Working group, Chair of committee, mentoring; value and foster diversity


%[alignment w/ culture: reference my understanding of company values and how they align with my work style] 
	My proficiencies in statistics and as a leader have been recognized in different ways. Most recently, two internal NIST Building the Future grants were awarded to me and my co-authors for an exploration into standards for deepfake detection and for advancing statistical methods in clock metrology. While at Swarthmore, I was nominated by the ASA to join the National Advisory Committee to the US Census and as a graduate student, I received awards for my dissertation work. I look forward to the opportunity to grow my expertise through contributing to cutting-edge developments in statistical data privacy.
%[key achievements: brief mention of specific professional achievements, awards, recognitions, quantify accomplishments where possible ]
%coauthored 2 successful internal grant proposals at NIST, NAC service  
%values: Collaboration, Equity, Inclusivity, Independence, and Integrity


	I am excited about the opportunity to begin a career at Urban and to expand my knowledge and skillset by working with the dedicated members of the Data Governance and Privacy Practice Area. Please let me know if you have any additional questions for me and feel free to view my professional website (link in footer below) for additional information. Thank you for your time and consideration.\\
	
	\vspace{3mm}
	
	\makeletterclosing
	
\end{document}

