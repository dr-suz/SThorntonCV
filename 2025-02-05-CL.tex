
\documentclass[11pt,a4paper,sans]{moderncv}        

\moderncvstyle{casual}                           
\moderncvcolor{blue}                               % color options 'blue' (default), 'orange', 'green', 'red', 'purple', 'grey' and 'black'
%\renewcommand{\familydefault}{\sfdefault}         % to set the default font; use '\sfdefault' for the default sans serif font, '\rmdefault' for the default roman one, or any tex font name
%\nopagenumbers{}                                  % uncomment to suppress automatic page numbering for CVs longer than one page

% character encoding
\usepackage[utf8]{inputenc}                       % if you are not using xelatex ou lualatex, replace by the encoding you are using
%\usepackage{CJKutf8}                              % if you need to use CJK to typeset your resume in Chinese, Japanese or Korean

% adjust the page margins
\usepackage[scale=0.75]{geometry}
%\setlength{\hintscolumnwidth}{3cm}                % if you want to change the width of the column with the dates
%\setlength{\makecvtitlenamewidth}{10cm}           % for the 'classic' style, if you want to force the width allocated to your name and avoid line breaks. be careful though, the length is normally calculated to avoid any overlap with your personal info; use this at your own typographical risks...

% personal data
\name{Suzanne}{Thornton, PhD}
%\title{Rutgers University}                               % optional, remove / comment the line if not wanted
\address{Philadelphia, PA}{USA}% optional, remove / comment the line if not wanted; the "postcode city" and and "country" arguments can be omitted or provided empty
\phone[mobile]{+1~(863)~370~9389}                   % optional, remove / comment the line if not wanted
%\phone[fixed]{+2~(345)~678~901}                    % optional, remove / comment the line if not wanted
%\phone[fax]{+3~(456)~789~012}                      % optional, remove / comment the line if not wanted
\email{thornton.suzy@gmail.com}                               % optional, remove / comment the line if not wanted
\homepage{dr-suz.github.io/}                         % optional, remove / comment the line if not wanted
%\extrainfo{additional information}                 % optional, remove / comment the line if not wanted
%\photo[64pt][0.4pt]{picture}                       % optional, remove / comment the line if not wanted; '64pt' is the height the picture must be resized to, 0.4pt is the thickness of the frame around it (put it to 0pt for no frame) and 'picture' is the name of the picture file
%\quote{Some quote}                                 % optional, remove / comment the line if not wanted

% to show numerical labels in the bibliography (default is to show no labels); only useful if you make citations in your resume
%\makeatletter
%\renewcommand*{\bibliographyitemlabel}{\@biblabel{\arabic{enumiv}}}
%\makeatother
%\renewcommand*{\bibliographyitemlabel}{[\arabic{enumiv}]}% CONSIDER REPLACING THE ABOVE BY THIS

% bibliography with mutiple entries
%\usepackage{multibib}
%\newcites{book,misc}{{Books},{Others}}
%----------------------------------------------------------------------------------
%            content
%----------------------------------------------------------------------------------
\begin{document}
	
	\recipient{Los Angeles Homeless Services Authority} 
	{}
	\date{\today }
	\opening{Dear Hiring Team,}
	\closing{Sincerely,}
	\makelettertitle
	
% The Los Angeles Homeless Services Authority (LAHSA) seeks motivated professionals who want to use their talents and skills to make a difference.  Our 750+ FTE staff are adaptive problem solvers and passionate about enriching people’s lives. If you are mission-driven, dedicated to superior service and support, and can diligently work independently in a collaborative environment, we would love for you to join our team. LAHSA is leading the fight to end homelessness in LA County.  Here, not only would your work have a real impact on the community, but we also offer a comprehensive and competitive benefits package.
%Created in 1993, LAHSA is a joint powers authority of the city and county of Los Angeles.  As the lead agency in the HUD-funded Los Angeles Continuum of Care, we coordinate and manage over $800 million annually in federal, state, county, and city funds for programs providing shelter, housing, and services to people experiencing homelessness.
%Under the supervision of the Data Systems Supervisor, this position is responsible to maintain, interpret, test, and analyze data to ensure information is being properly stored in the Los Angeles Continuum of Care Homeless Management Information System (LA CoC HMIS).

%The functions listed below are intended to describe the general nature and level of work being performed and are not to be interpreted as an exhaustive list of responsibilities. 
%Essential Job Functions
%Serve as the technical liaison between the HMIS participating organizations and vendor. 
%Provide technical support to HMIS participating organizations to ensure that data entered to HMIS is complete, accurate, and entered in a timely manner to meet reporting deadlines. 
%Analyze HMIS use and perform in-depth data validation on data from assigned participating organizations. 
%Identify, analyze, and interpret trends in complex data sets. 
%Identify and implement technical solutions where efficiencies may be gained through automation. 
%Develop and maintain queries, reports, and data feeds for internal and external users.
%Provide administration of implemented software, such as agency and user account set-up, system monitoring and testing, problem diagnosis and resolution routine software and information maintenance. 
%Conduct testing of software patches. 
%Assist senior staff and team members in assessing and improving processes.
%Provide feedback to the leadership team regarding changes or modifications to enhance the HMIS.
%Assist in assembling complex ideas, issues and observations into useful explanations and clear direction for the team.
%Provide technical assistance and cross training to other team members.
%Maintain knowledge on current HUD data elements and requirements. 
%Participate in meetings and business and data analysis activities with cross-functional teams to gather required reporting and dashboard requirements.
%Complete other duties as assigned.

%Knowledge, Skills & Abilities
%Intermediate or advanced knowledge of SQL/MySQL.
%Excellent people skills, specifically a capacity for collaboration and interpersonal relationships.
%Excellent written and verbal communication skills, including the ability to express technical concepts clearly to both technical and non-technical audiences.
%Strong organizational skills with a strong attention to detail.
%Expert problem solving, prioritization, and decision-making capabilities.
%Ability to prioritize multiple work assignments.
%Basic knowledge of data visualization tools (Tableau, Power BI, etc.).
%Advanced proficiency in MS Office Suite.

	
	My name is Suzanne Thornton and I am writing to apply for the Data Analyst (DMD09415) position at LAHSA. I realize that in some ways I may be different from the candidate you expect to hire. For instance, I hail from the east coast and I've worked primarily as a statistical researcher. Nevertheless, I hope you will consider my application because, in me, you will find a person of genuine compassion with a drive to use and develop their skills to improve upon the lives of people experiencing homelessness.    
	
	The commonalities between myself and people without shelter or homes has been apparent to me from a young age. The first time I really internalized the tenuous line between being housed and unhoused in our country was as a college student, volunteering at a food pantry. Gainesville, Florida is know as a local hub for people experiencing homelessness. As a college student living in scholarship housing and eventually moving up to renting a converted garage, I was aware of my privilege even as I pinched every penny. I participated in various volunteer activities that centered around providing food to the hungry. One day in a soup kitchen I remember starting up a conversation with a particularly grumpy young man I served. The 'Polyanna' in me wanted to offer some friendly company but I remember the deep dread I felt upon learning that this young man had a PhD. To me, education was my savior, the ticket to a solidly reliable middle-class life. In fact, my educational choices were driven by my desire to provide for myself in this way. But here was a stark reminder that no matter how educated, I am still vulnerable to the same circumstances as those experiencing homelessness. Over a decade later, I now live in West Philadelphia, on the outskirts of the city. I love my neighbors and I get to enjoy daily walks with my dog near a city park. But every day, we pass camps set up by people experiencing homelessness. Right along the city border, with plenty of wooded areas, there's one woman who has lived in a tent she built for herself for over five years now. I want to be of service to people who are in these kinds of situations. I want to put my talents and efforts into eradicating homelessness. I want to use statistics and data science for something worthwhile. 
	
	Finally, I want to address a weak point of my application. My experience with SQL is limited. Last year, I completed an introductory course on SQL on the online learning platform Coursera. However, I have not yet had the opportunity to use SQL in a professional capacity. I am a talented statistical programmer in other languages like R and Python and I am confident that I will lear and master SQL rapidly. Also, although my experience with Tableau is also limited (because I have primarily used R for data visualization), I have found the application to be very user friendly and I am looking forward to expanding my experience with this platform and documenting it HERE. 
	
	
	Please let me know if you have any additional questions for me and feel free to view my professional website (link in footer below) for additional information. Thank you for your time and consideration.\\
	
	\vspace{3mm}
	
	\makeletterclosing
	
\end{document}

