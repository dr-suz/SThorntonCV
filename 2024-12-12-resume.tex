%-------------------------
% Resume Template
% Author : Orest Tokovenko
% Github : https://github.com/oresttokovenko
% License : MIT
% Type: General Non-Technical
%
% Note: Compile with LuaLaTeX for Mac
%
%------------------------

%-------------------------------------------------------------------------------
%                Document & Class
%-------------------------------------------------------------------------------

\documentclass[letterpaper]{resume_config}

%\usepackage{lwarp}     %used to generate a html file
%\usepackage{color}
%\usepackage[usenames,dvipsnames]{xcolor}
%\usepackage{enumitem}
%\usepackage{fancyhdr}
%\usepackage{etoolbox,refcount}
%\usepackage{fontawesome5}
%\usepackage{geometry}
%\usepackage[hidelinks]{hyperref}
%\usepackage{fontspec}
%\usepackage[english]{babel}
%\usepackage{titlesec}
%\usepackage{multicol}
%\usepackage{pifont}
%\usepackage{calc}
%\usepackage{setspace}
%\usepackage{lipsum}
%
%
%\Preamble{xhtml}
%
%\Configure{textit}{\HCode{<em>}}{\HCode{</em>}}
%\Configure{textbf}{\HCode{<strong>}}{\HCode{</strong>}}
%\Configure{textsc}{\HCode{<span class="sc">}}{\HCode{</span>}}
%\Configure{$}{\HCode{<span class="inline-math">}}{\HCode{</span>}}{}
%\begin{document}
%\Css{.sc{font-variant: small-caps;}}
%\Css{.inline-math{font-style:italic;}}
%\EndPreamble

% NOTE: <- font size adjustment is available in the resume_config.cls file on the left hand side. ex. \LoadClass[#pt]{article} where # is 10, 11 or 12

% NOTE: If you need to adjust the space between any two lines, use this function: \vspace{#pt} where # is your number. ex. \vspace{-5pt} decreases the space between lines

%-------------------------------------------------------------------------------
%                Resume Starts Here
%-------------------------------------------------------------------------------

\begin{document}

\Header
    {Suzanne Thornton, PhD}  
    {Philadelphia, PA} 
    {863-370-9389}  
    {thornton.suzy@gmail.com} 
  %  {suzanne-thornton-0a535645} % linkedin [modified cls ]
    {dr-suz.github.io} \

%-------------------------------------------------------------------------------
%                Work Experience
%-------------------------------------------------------------------------------
\section{Summary}
Experienced professional statistical researcher transitioning to a profession of practice. Proven leadership skills. Strong statistical theorist and programmer with excellent communication abilities. %Longer CV including a complete publications list is available at: https://dr-suz.github.io/portfolio.html


\section{Work Experience}
\WorkExperience
    {National Institute of Standards and Technology PREP Research Scientist}  
    {George Washington University, Department of Engineering}  
    {Jan 2024 -- Present}  
    {Washington D.C.}  
    {
        \item Methods/models: Bayesian measurement error model, imputation for time series with gaps, text-based generative AI detection (theory)
   %     \item Software: R, R Markdown, TeX, terminal, git, Stan, Python, vscode  
   %     \item Platform: Mac, LINUX
        \item Other: Co-authored two (successful) NIST grant proposals 
    } 
%Formal Job description
%The mission of the School of Engineering and Applied Science (SEAS) of The George Washington University is to serve the global community by: providing high quality undergraduate, graduate and professional educational opportunities and stimulating and promoting innovative fundamental and applied research activities. This position is part of the National Institute of Standards (NIST) Professional Research Experience (PREP) program. NIST recognizes that its research staff may wish to collaborate with researchers at academic institutions on specific projects of mutual interest, thus requires that such institutions must be the recipient of a PREP award. The PREP program requires staff from a wide range of backgrounds to work on scientific research in many areas. Employees in this position will perform technical work that underpins the scientific research of the collaboration.

%The NIST Statistical Engineering Division seeks a researcher with a broad interest in statistical metrology to work on a variety of problems with NIST scientists, engineers, statisticians and other technical staff. Projects areas are likely to include research in the statistical characterization of nanomaterials to study chemical loading mechanisms in pharmaceutical applications (e.g., vaccine delivery) or assessing environmental contamination (e.g., contaminant adsorption on plastic nanoparticles), forensic science (e.g., analysis of DNA, footwear and tire tread, or other types of evidence), statistical methods for instrument calibration or the development of reference materials, characterization of semiconductor components or processes, or similar projects from a wide range of other physical science application areas. Statistical methods used may include experiment design, linear models, Bayesian modeling via Markov Chain Monte Carlo (MCMC), machine learning, or other techniques required to solve the problems at hand. Problems generally are collaborator-driven by the needs of the NIST technical staff in areas outside of statistics. Each project typically has a duration of several months to several years. Longer-term collaborative projects often have work that occurs in multiple-phases, however, and include publication of intermediate results.
%Working with NIST scientists, engineers, statisticians, and other technical staff to understand and precisely define relevant research questions for applications of interest
%Designing experiments using principles of statistical experiment design, as needed, to answer relevant scientific research questions formulated with collaborators
%Preparing data for analysis, as needed, with an emphasis on reproducible data preprocessing pipelines for data sets requiring a significant level of preparation
%Analyzing data using both graphical methods and via statistical modeling fitting and inference using software tools and methods that support research reproducibility
%Developing software tools for analysis of data by other researchers either for specific projects or for specific computational methods (e.g., Shiny apps, R packages, etc.), as needed
%Presenting results at internal meetings and potentially to external stakeholders
%Ensuring that research results, protocols, software and documentation, or other work outputs have been shared with relevant NIST staff members or appropriately archived for future NIST use.


\WorkExperience
    {Assistant professor of statistics}  
    {Swarthmore College}  
    {Sept 2020 -- Dec 2023}  
    {Swarthmore, PA}  
    {
        \item Methods/models taught: mathematical statistics, regression, univariate analyses, data visualization         
 %       \item Software: R, R Markdown, TeX, terminal, github  
 %       \item Platform: Mac
        \item Other: Published/implemented ethical reasoning in introductory and advanced stats classes, Published in Philosophy of Science         
    } 

\WorkExperience
    {Special government employee}  
    {US Census Bureau National Advisory Committee on Racial, Ethnic, and Other Populations}  
    {Aug 2022 -- Dec 2023}  
    {Washington, D.C.}  
    {
        \item Methods/models: Government statistics, survey data strategy and implementation
        \item Other:  Interdisciplinary collaboration and engagement with user feedback 
    } 

%\WorkExperience
%    {Online Workshop Facilitator} % job title
%    {University of Texas at Austin Charles A. Dana Center} % company name
%    {Summer 2020-2022} 
%    {Remote} 
%    {
%        \item Led intensive, online six week workshop for statistics educators without formal statistical training.
%    } 

\WorkExperience
    {Visiting assistant professor of statistics}  
    {Swarthmore College}  
    {Aug 2019 -- Aug 2020}  
    {Swarthmore, PA}  
    {
        \item Methods/models taught: regression, univariate analyses, data visualization  
%        \item Software: R, R Markdown, TeX 
%        \item Platform: Mac  
        \item Other: Designed and taught successful hybrid statistics courses 
        } 


\WorkExperience
    {Statistical consultant}  
    {Office of Statistical Consulting, Rutgers University} 
    {Sept 2016 -- Aug 2019} 
    {New Brunswick, NJ}  
    {
        \item Methods/models: Predictive modeling, case-control studies, cross validation, exact inference, goodness-of-fit, bootstrap optimism-corrected AUC  
   %     \item Software: R, Word 
     %   \item Platform: Mac, Windows
        \item Other: Co-authored publication in {\it Neurology} with MD from at Robert Wood Johnson Hospital
     %   \item  Development and validation of a predictive model of drug-resistant genetic generalized epilepsy 
    } 


%-------------------------------------------------------------------------------
%                Education
%-------------------------------------------------------------------------------

\section{Education}

\EducationExperience
    {Rutgers, The State University of New Jersey}  
    {Doctor of Philosophy in Statistics and Biostatistics} 
    {Oct 2019}  
    {New Brunswick, NJ}  
    {\item {\bf Thesis}: \href{https://rucore.libraries.rutgers.edu/rutgers-lib/61966/}{Advanced computing methods for statistical inference} 
     %\begin{itemize}
        \item[--] Methods/models: Approximate Bayesian computing, confidence distribution inference, algorithmic development, bootstrapping     
     %	\item[--] Software: R, TeX  
    % 	\item[--] Platform: Mac, Windows 
     	\item[--] Other: Parallel computing, taught SAS course   
     %	\end{itemize} 
     \item {\bf Publication}: \href{https://pubmed.ncbi.nlm.nih.gov/30430540/ }{Exact inference on the random-effects model for meta-analyses with few studies} 
     %\begin{itemize}
     	\item[--] Methods/models: Random effects, meta-analysis, exact (small sample) inference  
 %    	\item[--] Software: R, TeX
 %    	\item[--] Platform: Mac, Windows 
     %	\end{itemize} 
    }

\EducationExperience
    {University of Florida}  
    {Bachelor of Science in Mathematics and in Statistics}  
    {May 2014}  
    {Gainesville, FL}  
    {\item {\bf Thesis}: \href{https://ufdc.ufl.edu/AA00057032/00001}{Geometric ergodicity of Gibbs sampler for a hierarchical random effects model: Re-explained}
     \item[--] Methods/models: Markov chain Monte Carlo, Gibbs Sampling, Bayesian hierarchical random effects model     
%     \item[--] Software: TeX  
     \item[--]  Platform: Windows}

 
%-------------------------------------------------------------------------------
%                Skills
%-------------------------------------------------------------------------------
\section{Software and platforms}
\begin{SkillsList}
	\item Mac, Windows, Linux 
	\item R, RStudio, RMarkdown, Stan 
	\item Git, Github, GitLab 
	\item VScode, Microsoft Office
	\item LaTex
	\item Python 
\end{SkillsList}
%
%\section{Programming}
%\begin{SkillsList}
%    \item R\footnote{Expert} 
%    \item RMarkdown$^1$
%    \item SAS/STAT$^2$
%    \item Stan$^2$
%    \item Parallel computing$^2$
%\end{SkillsList}
%
%\section{Tools and Software}
%\begin{SkillsList}
%    \item Command~Line/Linux\footnote{Proficient} 
%    \item SQL\footnote{Advanced beginner} 
%    \item Tableau$^3$
%    \item Python$^3$
%    \item MS Office$^1$
%\end{SkillsList}
%
%\section{Other}
%\begin{SkillsList}
%    \item Statistical modeling$^1$
%    \item Data visualization/analysis$^1$ 
%    \item Academic writing$^1$
%    \item Non-technical writing$^1$
%    \item LateX$^1$ 
%    \item Git$^2$
%\end{SkillsList}


\end{document}

%-------------------------------------------------------------------------------
%                Resume Ends Here
%-------------------------------------------------------------------------------
