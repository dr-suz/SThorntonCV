
\documentclass[11pt,a4paper,sans]{moderncv}        

\moderncvstyle{casual}                           
\moderncvcolor{blue}                               % color options 'blue' (default), 'orange', 'green', 'red', 'purple', 'grey' and 'black'
%\renewcommand{\familydefault}{\sfdefault}         % to set the default font; use '\sfdefault' for the default sans serif font, '\rmdefault' for the default roman one, or any tex font name
%\nopagenumbers{}                                  % uncomment to suppress automatic page numbering for CVs longer than one page

% character encoding
\usepackage[utf8]{inputenc}                       % if you are not using xelatex ou lualatex, replace by the encoding you are using
%\usepackage{CJKutf8}                              % if you need to use CJK to typeset your resume in Chinese, Japanese or Korean

% adjust the page margins
\usepackage[scale=0.75]{geometry}
%\setlength{\hintscolumnwidth}{3cm}                % if you want to change the width of the column with the dates
%\setlength{\makecvtitlenamewidth}{10cm}           % for the 'classic' style, if you want to force the width allocated to your name and avoid line breaks. be careful though, the length is normally calculated to avoid any overlap with your personal info; use this at your own typographical risks...

% personal data
\name{Suzanne}{Thornton}
%\title{Rutgers University}                               % optional, remove / comment the line if not wanted
\address{Philadelphia, PA}{USA}% optional, remove / comment the line if not wanted; the "postcode city" and and "country" arguments can be omitted or provided empty
\phone[mobile]{+1~(863)~370~9389}                   % optional, remove / comment the line if not wanted
%\phone[fixed]{+2~(345)~678~901}                    % optional, remove / comment the line if not wanted
%\phone[fax]{+3~(456)~789~012}                      % optional, remove / comment the line if not wanted
\email{thornton.suzy@gmail.com}                               % optional, remove / comment the line if not wanted
\homepage{dr-suz.github.io/}                         % optional, remove / comment the line if not wanted
%\extrainfo{additional information}                 % optional, remove / comment the line if not wanted
%\photo[64pt][0.4pt]{picture}                       % optional, remove / comment the line if not wanted; '64pt' is the height the picture must be resized to, 0.4pt is the thickness of the frame around it (put it to 0pt for no frame) and 'picture' is the name of the picture file
%\quote{Some quote}                                 % optional, remove / comment the line if not wanted

% to show numerical labels in the bibliography (default is to show no labels); only useful if you make citations in your resume
%\makeatletter
%\renewcommand*{\bibliographyitemlabel}{\@biblabel{\arabic{enumiv}}}
%\makeatother
%\renewcommand*{\bibliographyitemlabel}{[\arabic{enumiv}]}% CONSIDER REPLACING THE ABOVE BY THIS

% bibliography with mutiple entries
%\usepackage{multibib}
%\newcites{book,misc}{{Books},{Others}}
%----------------------------------------------------------------------------------
%            content
%----------------------------------------------------------------------------------
\begin{document}
	
	\recipient{National Institute of Standards}
	{Professional Research Experience Program\\
	Statistical Engineering Division}
	\date{\today }
	\opening{To whom it may concern,}
	\closing{Sincerely,}
	\makelettertitle
	
	
	My name is Suzanne Thornton and I am writing to apply for the Sr. Research Data Analyst position (117379). Over the last year, I have enjoyed working as an affiliate of the National Institute for Standards of Technology (NIST) and collaborating on statistical research with the Statistical Engineering Division. I transitioned into this role after spending a few years teaching college-level statistics and finding myself most motivated and content when applying statistical theory and programming to real-world problems. I am especially eager to foster a career of practice that is oriented towards contributions in public health. 
	
	My interest in public health applications reaches back to my time as a graduate student where I collaborated with a doctor from the Robert Wood Johnson hospital to develop a predictive model for drug-resistant epilepsy. As the sole statistician on this project, I was responsible for data cleaning and exploration, model building and evaluation, and drawing proper conclusions from the data and model. This collaborative experience in developing data-driven solutions to a pressing medical issue resulted in a publication in \emph{Neurology}. After graduating, I published my dissertation research on an advanced computational inference technique called approximate confidence distribution computing while teaching at Swarthmore College. A NIST, I have enjoyed the opportunity to work on scientific problems related to the analysis of high precision data. Currently, I am constructing and evaluating a novel Bayesian measurement error model to more accurately represent the relationship between nanoparticle size and loading. I am also collaborating with an interdisciplinary group of statisticians and clock scientists on establishing guidelines for the analysis of atomic clock data. 
		
	My proficiencies in statistics and as a leader have been recognized in different ways. Most recently, two internal NIST Building the Future grants were awarded to me and my co-authors for an exploration into standards for deepfake detection and for advancing statistical methods in clock metrology. While at Swarthmore, I was nominated by the American Statistical Association to serve on the National Advisory Committee to the US Census. In 2020, I was selected by the president of the American Statistical Association (ASA) to lead a working group on LGBTQ+ inclusion within the discipline and last year I was the chair of the LGBTQ+ Advocacy Committee. My priorities as a leader in these positions have been to foster an inclusive, welcoming environment where people feel empowered to contribute their perspectives and to explore their curiosity and their passions. My most enjoyable professional experiences thus far have been in interdisciplinary team settings where a diverse group of individuals work together towards shared goals. I have received positive feedback from these endeavors, not the least of which was overwhelmingly positive student feedback in the Spring semester of 2020, when our college courses abruptly switched to a virtual format. As an instructor, I also developed coursework that integrated the American Statistical Association's Guidelines for Ethical Practice and have several publications on ethical considerations for data collection and analysis.
		
	\vspace{2mm} 
	
	I am excited about the opportunity to continue to expand my career in applying statistical tools to the important problems in public health. Please let me know if you have any additional questions for me and feel free to view my professional website (link in footer) for additional information. Thank you for your time and consideration.\\
	
	\vspace{3mm}
	
	\makeletterclosing
	
\end{document}

