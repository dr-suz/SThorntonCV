\documentclass[12pt]{letter}
\usepackage[a4paper,margin=1in]{geometry}
\usepackage{hyperref}

\begin{document}

\begin{letter}{}

\address{\parbox[t]{\textwidth}{
2024 Foundation for Individual Rights and Expression\\
P.O. Box 40128 \\
Philadelphia, PA 19106}}


\opening{Dear Chief People Officer Cait Scanlan,}

I am writing to apply for the Data Analyst position at FIRE in Philadelphia, PA. With a strong background in statistical research, data analysis, and communication, I am eager to contribute my expertise in support of our civil liberty of freedom of speech and thought. 

%[My experience as a Research Scientist at the National Institute of Standards and Technology has provided me with expertise in Bayesian measurement error models, time series imputation, and generative AI detection, equipping me with the analytical skills necessary for managing complex health datasets. Additionally, my tenure as a statistical consultant at Rutgers University has strengthened my ability to develop predictive models, conduct cross-validation analyses, and apply statistical inference methods, all of which are critical to analyzing administrative claims data and synthesizing substance-use-related datasets from diverse sources. My background in education has further honed my ability to communicate complex statistical concepts clearly and effectively, both in writing and verbally, making technical findings accessible to diverse audiences.] 
%
%[Beyond technical proficiency, I have extensive experience in interdisciplinary collaboration and data communication. As a Special Government Employee on the U.S. Census Bureau's National Advisory Committee, I engaged in government statistics and survey data strategies, ensuring methodological rigor while incorporating user feedback. Another successful collaborative experience with medical practitioners from Robert Wood Johnson Hospital resulted in a \emph{Neurology} journal article. My ability to translate statistical findings into actionable insights will enable me to provide clear, data-informed recommendations to PDPH and DBHIDS leadership.]
%
%connections and experience with federal data science and stats 
%work with complicated data and models 
%skill of programming, in R in particular. coding is my favorite part of my professional expertise.  
%advising students who worked on data scraping final projects for classes i taught 
%communication and visualization 
%academic writing and interdisciplinary collaboration 
%taught regression and data processing (cleaning and preparation)
%nist work on deep fake detection, 
   
As a professional statistical researcher and educator, my experience includes working with complex data and models, communicating data analysis to general audiences (including visualizations), academic writing, interdisciplinary collaboration, and extensive statistical programming. I am particularly interested in the careful study of data that is scraped from the web. In this position, one of my personal goals will be to create readable code and design scalable, transparent data scraping and analysis tools. In this dawn of the age of AI, I believe my professional tasks in this position will require both the freedom to explore and test machine learning tools while still encouraging criticism of and maintaining standards for them.

One of my proudest achievements thus far has been a collaborative article bringing to light the need for more careful consideration of certain qualitative variable definitions for human subjects (https://dr-suz.github.io/Stat11/Towards_statistical_best_practices.pdf). This publication created professional bonds that have supported me since grad school, and it relates to a topic that is deeply important to me. As expected, this article met with resistance; although, disappointingly, the challenger (https://dr-suz.github.io/Stat11/a_big_ask.pdf) was not one of academic rigor, nor did it seem to be made in good faith. (If you examine the URL, you will see that this article is linked from my personal webpage in the material for an introductory Statistics class. It was important to me to present both articles to my students and to encourage them to critique both analyses for themselves.) I wrote a letter to the editor in response to this rebuttal, and, although the UK-based stats magazine gave me a bit of a run around before ultimately refusing to publish my letter(https://www.academia.edu/108959594/Letter_to_the_Editor_of_Significance), I am assuaged by the freedom of communication I have access to via the internet. I have posted my letter (and an appendix https://www.academia.edu/106176568/Appendix_to_Letter_to_the_Editor_of_Significance) where others can view my critiques of this article freely. I mention this story to highlight an example of the importance of my First Amendment right that overlaps with both my personal and professional lives. It is crucial to me to protect this right for myself and others.    

I am particularly drawn to this opportunity because my professional experiences thus far have made it clear to me that I am primarily a values-motivated individual. I am ready to move towards the next step of my career that aligns my knowledge, skills, and abilities with my values. The chance to contribute my expertise in defense of one of our fundamental civil liberties is far more motivating to me than prioritizing contributing to esoteric statistical research. 

I welcome the opportunity to discuss how my skills and experience align with your needs. Additional demonstrations of my work can be found on my personal website (https://dr-suz.github.io/portfolio.html) and include both technical and non-technical publications, data visualizations, and educational resources of my own creation. Thank you for your time and consideration.


\noindent Sincerely,\\ \vspace{2ex} 
\noindent Suzanne Thornton, PhD


\end{letter}

\end{document}

Previous life experiences where I felt stifled and unable to express myself have culminated in my professional life in a commitment to academic and scientific rigor, which I believe can only be upheld with a societal dedication to the freedom of speech. 