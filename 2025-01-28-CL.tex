
\documentclass[11pt,a4paper,sans]{moderncv}        

\moderncvstyle{casual}                           
\moderncvcolor{blue}                               % color options 'blue' (default), 'orange', 'green', 'red', 'purple', 'grey' and 'black'
%\renewcommand{\familydefault}{\sfdefault}         % to set the default font; use '\sfdefault' for the default sans serif font, '\rmdefault' for the default roman one, or any tex font name
%\nopagenumbers{}                                  % uncomment to suppress automatic page numbering for CVs longer than one page

% character encoding
\usepackage[utf8]{inputenc}                       % if you are not using xelatex ou lualatex, replace by the encoding you are using
%\usepackage{CJKutf8}                              % if you need to use CJK to typeset your resume in Chinese, Japanese or Korean

% adjust the page margins
\usepackage[scale=0.75]{geometry}
%\setlength{\hintscolumnwidth}{3cm}                % if you want to change the width of the column with the dates
%\setlength{\makecvtitlenamewidth}{10cm}           % for the 'classic' style, if you want to force the width allocated to your name and avoid line breaks. be careful though, the length is normally calculated to avoid any overlap with your personal info; use this at your own typographical risks...

% personal data
\name{Suzanne}{Thornton}
%\title{Rutgers University}                               % optional, remove / comment the line if not wanted
\address{Philadelphia, PA}{USA}% optional, remove / comment the line if not wanted; the "postcode city" and and "country" arguments can be omitted or provided empty
\phone[mobile]{+1~(863)~370~9389}                   % optional, remove / comment the line if not wanted
%\phone[fixed]{+2~(345)~678~901}                    % optional, remove / comment the line if not wanted
%\phone[fax]{+3~(456)~789~012}                      % optional, remove / comment the line if not wanted
\email{thornton.suzy@gmail.com}                               % optional, remove / comment the line if not wanted
\homepage{dr-suz.github.io/}                         % optional, remove / comment the line if not wanted
%\extrainfo{additional information}                 % optional, remove / comment the line if not wanted
%\photo[64pt][0.4pt]{picture}                       % optional, remove / comment the line if not wanted; '64pt' is the height the picture must be resized to, 0.4pt is the thickness of the frame around it (put it to 0pt for no frame) and 'picture' is the name of the picture file
%\quote{Some quote}                                 % optional, remove / comment the line if not wanted

% to show numerical labels in the bibliography (default is to show no labels); only useful if you make citations in your resume
%\makeatletter
%\renewcommand*{\bibliographyitemlabel}{\@biblabel{\arabic{enumiv}}}
%\makeatother
%\renewcommand*{\bibliographyitemlabel}{[\arabic{enumiv}]}% CONSIDER REPLACING THE ABOVE BY THIS

% bibliography with mutiple entries
%\usepackage{multibib}
%\newcites{book,misc}{{Books},{Others}}
%----------------------------------------------------------------------------------
%            content
%----------------------------------------------------------------------------------
\begin{document}
	
	\recipient{Pinterest}  
	{}
	\date{\today }
	\opening{Dear Hiring Team,}
	\closing{Sincerely,}
	\makelettertitle
	
	I am an experienced statistical researcher and educator who wants to transition into a profession of practice. In this letter, I will clarify my background with respect to the stated job requirements for a data scientist position. As an avid Pinterest user, I am excited about the opportunity to contribute to the development of this popular platform for creativity and community. 
	
	Over the past five years, I have worked as a part-time statistical consultant, a college-level educator/statistics researcher, and an applied research scientist. In each of these positions, I was required to meet time-sensitive deadlines involving data analysis with little-to-no oversight or management. The real-world problems I've contributed solutions to include predictive medical models and scientific models for precise measurement science applications in nanotechnology and clocks. Furthermore, as a professor of statistics, I provided guidance to students working on class projects involving data collection and regression analysis which sometimes included supporting students with web-scraping. Currently, in my remote position with the National Institute of Standards and Technology, I am working through the book {\em Deep Learning with R} to gain familiarity and practice with the deep learning framework TensorFlow. The common element among my current and prior work experiences is a mastery of analytical problem solving with statistical modeling techniques, including machine learning methods like regression. 
	 
	I am an expert in the statistical scripting language R and I am proficient with Python. I have not yet had the opportunity to work in SQL but I am familiar with this database language having studied it independently. Although I am lacking in experience with large, complex, web-content data, my research and applied statistical contributions have involved large data sets (e.g. hundreds of thousands of observations) and data generated from complex models (e.g. measurement error models and computational models). I am intrigued by modern data science challenges regarding large, high dimensional, complex data and will prove to be an eager learner, given the opportunity to excel in this area. 
	
		Compared to other candidates, I believe my communication skills and ability to work well with a varied team of collaborators are what set me apart. I have been heavily involved in leadership positions within the American Statistical Association since 2020 when I was selected by the (then) president to lead a working group on LGBTQ+ inclusion within the discipline. I appreciate working within teams who value an inclusive, welcoming environment where people feel empowered to contribute their perspectives. My most enjoyable work experiences thus far have been in interdisciplinary team settings where a diverse group of individuals work together towards shared goals. I am skilled at communicating complex topics to both technical and non-technical audiences as evidenced by my publication and reviewer record with academic journals and by my experience teaching undergraduate statistics students.
%		I have received positive feedback from these endeavors, not the least of which was overwhelmingly positive student feedback in the Spring semester of 2020, when our college courses abruptly switched to a virtual format.   

	
	%\vspace{-3mm}
	
	\makeletterclosing
	
\end{document}
